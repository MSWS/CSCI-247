\documentclass{article}

\usepackage{enumitem}
\usepackage{amsmath, amssymb}
\usepackage{geometry}

\geometry{tmargin=1in, bmargin=1in, lmargin=1in, rmargin=1in}

\begin{document}

\title{CSCI 247}
\author{Isaac Boaz}
\maketitle

\section*{Question 1}
Given a floating-point format with a \(k\)-bit exponent and an \(n\) bit fraction, write formulas for the exponent \(E\), the significand \(M\), the fraction \(f\), and the value \(V\) for the quantities that follow. In addition, describe the bit representation.

\begin{enumerate}[label=\Alph*)]
    \item \underline{The number \(7.0\)} \\[\baselineskip]
    Keeping in mind the general formula 
    \[V = (-1)^s \cdot M \cdot 2^E\]
    \begin{tabular}{rl}
        where & \(V\) is the value, \\
        & \(s\) is the sign, \\
        & \(M\) is the mantissa, \\
        & and \(E\) is the exponent. \\
    \end{tabular}

    \begin{enumerate}[label=\arabic*.]
        \item We'll first begin by converting 7.0 to its binary representation \\
        \(7_{10} = 111_2\)
        \item Normalizing this representation leaves us with
        \[111_2 = 1.11 \cdot 2^2\]
        \item Meaning the exponent \(Exp = 2\)
        \item And the fractional part is \(f = 111^*\) \\
        Note: IEEE implies a leading \(1\) at the beginning, so \(f = 110\)
        \item Calculate the biasd \(E\) by adding \(Bias = 2^{k-1} - 1\)
        \[E = 2 + (2^{k-1} - 1) = 2^{k-1} + 1\]
        \item Combing with an example of \(k = 4, M = 3\)
        \begin{align*}
            Exp =&\ 2 \\
            E =&\ 2^{4-1} + 1 = 9_{10} = 1001_2 \\
            M =&\ 1.11_2 \\
            f =&\ 110_2 \\
            V =&\ (-1)^s \cdot (1.)f \cdot 2^{Exp} \\
              =&\ (1.)11_2 \cdot 2^{2} = 7.0
        \end{align*}
        With a final bit representation of
        \begin{tabular}{r|c|l}
            Sign & Exp & Mantissa \\
            0 & 1001 & 110 \\
        \end{tabular}
    \end{enumerate}

    \pagebreak

    \item \underline{The largest odd integer that can be represented exactly}
    \begin{itemize}
        \item We know the sign must be 0 for positives.
        \item The mantissa should be all 1's to get the largest representable odd number.
        \item If our exponent goes beyond the mantissa's accuracy, we'll end up with a 0 as the least significant bit (IE an even number).
        \item Thus, our exponent is \(Exp = Min(k, M)\)
    \end{itemize}
    An example where \(k = 4, M = 3\)
    \begin{align*}
        Exp =&\ 3 \\
        Bias =&\ 2^{k-1} - 1 = 2^{4-1} - 1 = 7 \\
        E =&\ Exp + Bias \\
          =&\ 3 + 7 = 10_{10} = 1010_2 \\
        f =&\ 111_2 \\
        V =&\ (1.)111_2 \cdot 2^{3} = 15
    \end{align*}
    With a final bit representation of
    \begin{tabular}{r|c|l}
        Sign & Exp & Mantissa \\
        0 & 1010 & 111 \\
    \end{tabular} \\
    \pagebreak

    \item \underline{The reciprocal of the smallest positive normalized value} \\ [\baselineskip]
    In order to find the reciprocal of any number, we can simply "flip" the numerator and denominator. This can be achieved in floating point representation by multiplying the exponent by -1. \\
    The smallest possible value given \(k\) exponent bits and \(M\) fraction bits is when:
    \begin{itemize}[label=\(\square\)]
        \item \(M = 0\), \(s = 0\)
        \item \(Exp\) is the biggest positive value multiplied by \(-1\)
    \end{itemize}
    \begin{enumerate}[label=\arabic*.]
        \item Set \(M =0\), \(s = 0\)
        \item Calculate \(E = Exp - Bias\) with \(Exp = 1\)
        \item One example where \(k = 5, M = 4\)
        \begin{align*}
            Bias &= 2^{k-1} - 1 =\ 2^{5-1} - 1 &= 15 \\
            E &= (1 - Bias) =\ (1 - 15) \cdot -1 &= 14 \\
            Exp &= E + Bias =\ (14 + 15) &= 29_{10} = 11101_2 \\
            f &=& 0_2
        \end{align*}
        With a final bit representation of
        \begin{tabular}{r|c|l}
            Sign & Exp & Mantissa \\
            0 & 11101 & 000 \\
        \end{tabular} \\
        Which equals \(16384\)
    \end{enumerate}

\end{enumerate}

\end{document}